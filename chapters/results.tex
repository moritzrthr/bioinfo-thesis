\chapter{Results}
\label{ch:results}

\section{Model Training and Convergence}

The model was trained until the validation loss showed early signs of stagnation, indicating the onset of overfitting. Convergence was reached after 13 epochs, corresponding to approximately 1,550,000 training steps. To assess the final performance, the model state at this checkpoint was utilized for inference on the complete, independent test dataset.

\section{Performance Evaluation on the test set}

To evaluate the predictive confidence, precision-coverage curves were generated for the different PTM categories and stratified by TMT  vs non-TMT data.  The performance is quantified using the Area Under the Precision-Coverage Curve (AUPCC).

The evaluation follows a peptide-centric approach: a single ground truth peptide contributes to the curves of all modifications it contains. For instance, a peptide sequence such as \enquote{[+229.997]-PEPT[+79.966]IDEK[+14.016]} is included in both the precision-coverage curve for phosphorylated threonine (T[+79.966]) and monomethylated lysine (K[+14.016]). In all subsequent analyses, the performance on unmodified peptides (light grey in Figure \ref{fig:PCfacets}) serves as a reference baseline.

\begin{figure}[htbp]
    \centering
    \includegraphics[width=0.8\textwidth]{plots/results/PCfacets.png}
    \caption{Precision-coverage curves at the peptide level across different PTMs. PTM types (Acetylation, Citrullination (Citrull\_deamid), Monomethylation, OGalNAc/OGlcNAc (O-Glyco), Oxidation, Phosphorylation, Pyro-glu, and Ubiquitination) are shown in the different panels. Colors represent the performance of the final model on TMT-labeled (orange) and non-TMT (blue) spectra. Unmodified peptides are shown in light grey for comparison.}
    \label{fig:PCfacets}
\end{figure}

\subsection{Comparative Performance: Non-TMT and Modanovo Baseline}

A key requirement for the expanded model was to maintain high performance on standard (non-TMT) data, ensuring that the addition of TMT-related features did not degrade general sequencing accuracy. On non-TMT unmodified peptides, the model achieved an AUPCC of 0.92, which is highly consistent with the performance reported for the Modanovo architecture (AUPCC of 0.93) \cite{Yilmaz2022}.

This stability is further reflected in the specific PTM performance. For instance, the model demonstrated similar accuracy for Phosphorylation on non-TMT data (AUPCC 0.95) staying aligned with the general Modanovo median for phosphorylation (0.93--0.96 depending on the residue) \cite{Yilmaz2022}. Similarly, high performance was observed for Pyro-Glu (AUPCC 0.95), likely due to its restricted occurrence at the peptide N-terminus, a pattern the model successfully internalizes. In contrast, O-Glycosylation (S/T[+203.079]) remains a significant challenge with an AUPCC of 0.39, mirroring the difficulties reported in literature due to low training examples and complex fragmentation \cite{Yilmaz2022}.

\subsection{Generalization on TMT-Labeled  Data}

The global performance on TMT data (AUPCC 0.89) is slightly lower than on non-TMT data (AUPCC 0.92), reflecting the increased spectral complexity introduced by the chemical labels.

Specific findings for TMT-labeled PTMs include:
\begin{itemize}
    \item \textbf{Ubiquitination and Monomethylation:} Interestingly, the model performed slightly better on TMT-labeled Ubiquitination (AUPCC 0.95) compared to its non-TMT counterpart (AUPCC 0.94). For Monomethylation, the gap was even more pronounced (TMT: 0.94 vs. non-TMT: 0.82), suggesting that the systematic mass shifts provided by TMT might aid in distinguishing these specific modifications in certain contexts.
    \item \textbf{Acetylation:} Performance remained robust across both categories (TMT: 0.91 vs. non-TMT: 0.92), indicating that Acetylation is identified with high confidence regardless of the labeling strategy.
    \item \textbf{Oxidation:} A noticeable decrease in performance was observed for TMT-labeled oxidized peptides (AUPCC 0.81) compared to non-TMT (AUPCC 0.90), pointing towards potential interference between the TMT-label fragments and the neutral losses associated with methionine oxidation.
\end{itemize}

The  final precision at full coverage (e.g., 0.59 on all tmt-labeled spectra) demonstrates that the model provides a reliable foundation for uncovering biological insights in multiplexed quantitative proteomics experiments.


\section{Hyperparameter Optimization and Isotopic Sensitivity}

A critical challenge identified during the evaluation of TMT-labeled data was the increased frequency of non-monoisotopic precursor selection. In TMT multiplexed samples, the high density of ions and the chemical labeling itself often lead to a shift where the instrument triggers fragmentation on an isotopic peak rather than the monoisotopic mass. To address this, the model configuration was iteratively optimized.

\subsection{Isotope Error Range and Beam Search Synergy}

The initial baseline configuration (Config0) utilized a restrictive isotope error range of $0,1$, which often forced the model to strictly adhere to the theoretical monoisotopic mass, leading to sequence errors when the precursor mass was incorrectly assigned by the instrument. By expanding this range to $0,3$ (Config ISO), the model gained the necessary flexibility to account for these systematic mass shifts without being penalized for "off-by-one" or "off-by-two" Dalton errors.

Furthermore, while literature for models like Casanovo suggests that increasing the beam size has diminishing returns, our results indicate a strong synergy between isotope flexibility and search breadth. Increasing the beam size from 1 to 5 (Config FINAL) allowed the decoder to explore a wider sequence space, which proved essential for resolving the complex fragmentation patterns of TMT-labeled and modified peptides.

\subsection{Quantitative Impact of Configuration Changes}

The optimization process led to a stepwise improvement in all global metrics (see Table \ref{tab:config_comparison}). The global AUPCC increased from 0.8746 (Config0) to 0.8817 (ISO) and reached its peak at 0.8988 in the FINAL configuration. Notably, the most significant breakthrough occurred in the jump to the FINAL configuration, where the global precision at full coverage rose from 0.5726 to 0.6146.

\begin{table}[htbp]
\centering
\caption{Comparison of model performance across different configurations.}
\label{tab:config_comparison}
\begin{tabular}{lccc}
\hline
\textbf{Metric} & \textbf{Config0} & \textbf{Config ISO} & \textbf{Config FINAL} \\ \hline
Global AUPCC    & 0.8746           & 0.8817              & 0.8988                \\
Global Precision (Cov 1.0) & 0.5726 & 0.5726              & 0.6146                \\ \hline
\end{tabular}
\end{table}

\subsection{Impact on Specific Modification Classes}

The transition to Config FINAL particularly benefited complex PTMs:

\begin{itemize}
    \item \textbf{Ubiquitination:} This category showed a massive precision jump from 0.6675 to 0.7359. The combination of expanded isotope ranges and increased beam size likely allowed the model to better identify the characteristic GlyGly-remnant fragmentation patterns, which are often obscured in TMT-labeled spectra.
    \item \textbf{Phosphorylation:} As the largest dataset (N=293,256), the shift from 0.8556 (Config0) to 0.8784 (AUPCC) and a precision increase to 0.5656 is statistically the most significant indicator of the model's enhanced robustness.
    \item \textbf{Monomethylation:} The AUPCC rose from 0.9032 to 0.9326. Remarkably, TMT-labeled methylated peptides achieved a significantly higher AUPCC (0.9443) than their non-TMT counterparts (0.8211). This suggests that the model learned to leverage the TMT-induced mass shifts as a "fingerprint" to distinguish methylation from isobaric interferences more effectively than in unlabeled data.
\end{itemize}

The improvement in Global TMT AUPCC (from 0.8640 to 0.8717 in the ISO stage) confirms that accounting for isotopic uncertainty is a prerequisite for high-confidence de novo sequencing in multiplexed workflows.





\section{Application on independent Glioma TMT Dataset}

To evaluate the model's discovery potential, inference was performed on the complete Glioma TMT dataset, comprising 6.48 million spectra. Given that \textit{de novo} sequencing operates independently of protein databases, it is crucial to validate the predicted sequences against a reference proteome to distinguish between high-confidence identifications and potential false positives.

\subsection{Proteome Alignment and Score Calibration}

The predicted sequences for unmodified peptides were aligned against the human reference proteome using \texttt{blastp}. This alignment process serves as a first-tier validation of the model's output quality. The sequences were categorized based on their alignment criteria: perfect matches (identity = 100\%), sequences with one mismatch, and sequences with two mismatches.

Out of the total predictions, 1.34 million peptides aligned perfectly with the reference proteome, while 0.62 million and 0.8 million sequences exhibited one and two mismatches, respectively.


\begin{figure}[htbp]
    \centering
    \includegraphics[width=0.80\textwidth]{plots/results/DeNovo_Glioma_TMT_Results.png}
    \caption{Cumulative alignment of \textit{de novo} predictions to the human reference proteome across confidence score thresholds. The plot illustrates the proportion of sequences matching the reference with zero (blue), up to one (orange), and up to two (green) amino acid substitutions. The segments between the curves represent peptides with specific mismatch counts, potentially accounting for Single Nucleotide Polymorphisms (SNPs) or technical sequencing errors. The increasing trend demonstrates the correlation between the model's self-reported confidence and sequence accuracy.}
    \label{fig:alignment_results}
\end{figure}

The upward trend of the  bars in Figure \ref{fig:alignment_results} confirms that the internal scoring mechanism of the Transformer architecture effectively reflects the probability of a sequence being biologically "correct." At the highest confidence bin ($>0.95$), the vast majority of sequences are plausibel in sense that they appear in the proteome, providing a solid baseline for the subsequent analysis of modified and novel peptides.

\subsection{Stratified Performance and Modification Stability}

To further investigate the reliability across different chemical states, the precision of the predictions was stratified by modification type and ranked by confidence (PSM Rank). In this analysis, the "Rank 1" prediction represent the highest-confidence spectrum.

\begin{figure}[htbp]
    \centering
    \includegraphics[width=0.85\textwidth]{plots/results/rank_precision_offby0_stratified.png}
    \caption{Proportion of perfectly aligned PSMs stratified by modification type and ranked by confidence score.}
    \caption{Stratified precision-rank analysis of the adpated model. The plot shows the proportion of Peptide-Spectrum Matches (PSMs) achieving a perfect \textit{blastp} alignment to the human reference proteome, ranked by descending model confidence scores. Curves are stratified by modification type: unmodified or just oxidation/acetylation(green), phosphorylation (red), monomethylation (blue), and other PTMs (purple). Symbols represent specific confidence score thresholds ($\geq 0.80, 0.90, 0.95$). While performance remains high ($>90\%$) for most classes at high confidence, monomethylated peptides show a significant precision drop, indicating potential challenges in mass-equivalent deconvolution or model overconfidence for this specific modification.}
    \label{fig:rank_precision}
\end{figure}

As shown in Figure \ref{fig:rank_precision}, most modifications follow the expected trend where decreasing confidence ranks correlate with a lower proportion of perfect alignments. Phosphorylated peptides follow the global trend closely.

A notable outlier is Monomethylation. While this modification showed excellent performance in the AUPCC metrics on the test set, the proportion of perfect alignments drops more sharply with increasing rank compared to other modifications. This discrepancy suggests the absolute mass shifts or the localization of the methyl group might still lead to mismatches during genomic alignment in real-world samples.

\subsection{Summary of Validation}

The alignment results demonstrate that the expanded model produces biologically plausible peptide sequences. The high correlation between the model's confidence score and the genomic match rate validates the use of these scores as a filter for discovery. This foundation allows for the exploration of spectra that remain unidentified by traditional database-driven methods, such as MaxQuant, which will be discussed in the following section.


\section{Uncovering the Dark Proteome: Comparison with MaxQuant}

A primary objective of this study was to evaluate the extent to which \textit{de novo} peptide sequencing can expand the identification landscape beyond the limitations of database-driven methods. By comparing our results with those obtained via MaxQuant (MQ), we can quantify the "dark matter" of the proteome—spectra that contain high-quality information but remain unidentified by traditional search engines.

\subsection{Unique Peptide Identifications}

The comparison of unique peptide sequences identified by both methods reveals a substantial expansion of the detectable proteome. As shown in Figure \ref{fig:unique_peps}, the \textit{de novo} approach identified approximately 40,000 unique peptides that were also found by MaxQuant, demonstrating high consistency in the "known" space. 

\begin{figure}[htbp]
    \centering
    \includegraphics[width=0.7\textwidth]{plots/results/uniquePeps.png}
    \caption{Overlap of unique peptide identifications between the adpated \textit{de novo} model and MaxQuant. The model successfully recovers the majority of MQ identifications while contributing 40,000 additional unique sequences.}
    \label{fig:unique_peps}
\end{figure}

Crucially, the model discovered an additional 40,000 unique peptides that were entirely absent from the MQ results. Conversely, only a negligible fraction (approximately 30 peptides) was identified by MQ but missed by our model, indicating that the \textit{de novo} approach covers nearly the entire search space of the database-driven method while doubling the number of unique sequences.

\subsection{Identification of Previously Unidentified Spectra}

The true potential of the model is reflected at the spectrum level (PSMs). In the analyzed dataset, MaxQuant provided identifications for 1,510,866 spectra. Our model was able to provide high-confidence sequences for a significant portion of the remaining "dark" spectra:

\begin{itemize}
    \item \textbf{High-Confidence Matches (Score $> 0.8$):} We identified 446,290 additional spectra with a confidence score above 0.8. Based on our previous alignment validation, this score range corresponds to highly reliable peptide sequences.
    \item \textbf{Precursor-Shifted Matches (Score $-0.2 < s < 0$):} Interestingly, we found 231,208 spectra in a score range that indicates high sequencing confidence but carries a penalty ($-1$) due to a precursor mass mismatch. These spectra likely represent peptides with unexpected modifications or amino acid substitutions (SNPs) that shift the precursor mass beyond the tolerance of the theoretical database entry, yet yield clear fragmentation patterns.
\end{itemize}

Summing these categories, the \textit{de novo} approach provides plausible sequence candidates for over 670,000 spectra that were previously discarded in the MQ workflow. This substantial increase in spectral utilization highlights the model's ability to move beyond the "closed-search" paradigm.

\subsection{From Global Discovery to Specific Variants}

The identification of these additional sequences suggests that the "dark proteome" in these glioma samples is rich in biological variants. The high number of high-confidence predictions that do not perfectly match the database entries—especially those with slight precursor shifts—points towards the presence of non-canonical protein isoforms. In the following sections, we will categorize these findings into specific biological phenomena, namely novel phosphorylation sites (p-sites) and single nucleotide polymorphisms (SNPs).

\subsection{Discovery of SNP-Related Variants}

The discovery of over 670,000 previously unidentified spectra suggests that a significant portion of the "dark proteome" in these samples arises from sequences not represented in the canonical reference database. To investigate whether these unique identifications stem from biological mutations, we analyzed peptides that showed minor sequence deviations (one or two amino acids) compared to the closest database entry. To ensure a conservative and robust analysis, we employed a "clean-hit" filtering strategy: any peptide that produced a perfect match in any experimental fraction or database search was excluded, focusing our analysis solely on truly novel sequence candidates.

By aligning these unique sequences back to the human proteome, we evaluated whether the observed amino acid substitutions could be explained by Single Nucleotide Polymorphisms (SNPs). A substitution was flagged as a potential SNP if the transition between the canonical and the \textit{de novo} predicted amino acid could be achieved by a single nucleotide change within the corresponding codon \cite{Wang2011}. Our analysis revealed that approximately 30\% of all single amino acid deviations and 25\% of dual amino acid deviations are directly explainable by such genomic point mutations. For instance, we identified cases where the model predicted a sequence such as \textit{PAPTIT} instead of the canonical \textit{PEDTIT}. Both mismatches in this example—Glutamate (E) to Alanine (A) and Aspartate (D) to Threonine (T)—are reachable via a single nucleotide exchange in their respective codons (e.g., GAG $\rightarrow$ GCG). 

These findings demonstrate that the \textit{de novo} model effectively captures protein-level manifestations of genetic variability that remain invisible to standard "closed-search" workflows \cite{Alfaro2014}. The high percentage of explainable substitutions confirms that these are not random sequencing errors, but likely reflect the actual biological diversity of the glioma samples. This capability to identify non-canonical isoforms and mutations without requiring matched genomic data highlights the transformative potential of deep learning-based sequencing for personalized proteogenomics \cite{Choudhary2001}.

The identification of these amino acid substitutions at the protein level, however, does not inherently guarantee the presence of a corresponding genomic variant. To validate our findings, we integrated the \textit{de novo} results with genomic data from panel sequencing, covering 500 cancer driver genes. These genomic variants were annotated using the Ensembl Variant Effect Predictor (VEP) to identify potential Single Amino Acid Variants (SAAVs). By cross-referencing our data, we found that 694 identified SNPs were supported by direct genomic evidence, with the predicted SAAVs aligning with genomic mutations at high precision. In several instances, the evidence was further strengthened by multiple unique, overlapping peptides covering the same mutation site, providing independent proteomic confirmation for a single genomic event \cite{Wang2014}.

To ensure the spectral reliability of these novel identifications, we performed a validation based on spectral similarity. We compared the experimental spectra against theoretical fragment patterns using the spectral angle as a metric for similarity. This confirmed that the observed fragmentation matches the predicted pattern of the mutated sequence with high fidelity, whereas a comparison with the non-mutated, canonical sequence would result in significant mass shifts and poor spectral alignment, unless the substitution was isobaric (e.g., Leucine to Isoleucine) \cite{Gessulat2019}.

\begin{figure}[htbp]
    \centering
    \includegraphics[width=\textwidth]{plots/application/snpvalidationProsit.png}
    \caption{Left: Prosit-predicted spectrum (top) and experimental spectrum
(bottom) for the reference peptide sequence YAALLK. The cartoon
illustrates the relevant nucleotide sequence and fragment ion series assuming the reference genome allele. Right: Same as for left, but for the peptide sequence
YAASLK predicted by our model for the same experimental spectrum. The higher degree of overlap between experimental and theoretical spectrum for the mutated peptide (right) validate the models predictipn.}
    \label{fig:snp_prosit_validation}
\end{figure}



 



The true strength of this proteogenomic approach lies in the context of TMT multiplexing. While database-driven searches often struggle with the increased complexity and altered fragmentation of labeled peptides, TMT-based workflows allow for the simultaneous quantification of these variants across multiple samples. By linking the identified SAAVs with the specific MS3 reporter ion channels, it becomes possible to map a mutation directly to a specific patient within a multiplexed run \cite{Pertosi2016}. This highlights the synergy of expanding \textit{de novo} sequencing to TMT data: it not only uncovers mutations beyond the reach of standard databases but also preserves the quantitative resolution necessary for clinical and biological interpretation in large-scale cohorts.

\subsection{Discovery of PTM Sites}


A key advantage of \textit{de novo} peptide sequencing is the ability to identify post-translational modifications (PTMs) without the inherent bias of a restricted search space. To evaluate the model's capacity for PTM discovery, we systematically analyzed predicted phosphorylation sites (p-sites) that were not identified by the MaxQuant (MQ) search.

The methodology for PTM mapping involved several steps: First, all \textit{de novo} predicted peptides were aligned against the human reference proteome. We considered two distinct alignment scenarios: (1) \textit{offby\_0}, representing perfect matches where the predicted sequence (including PTMs) matches the database entry exactly, and (2) \textit{offby\_1\_snp}, allowing for a single mismatch to account for potential single nucleotide polymorphisms or sequence variants. Following alignment, the predicted modification residues were mapped to their specific positions within the protein sequence. To ensure a conservative estimate, we filtered the results to unique p-sites, aggregating redundant spectral identifications to a single site per protein.

\begin{figure}[htbp]
    \centering
    \includegraphics[width=\textwidth]{plots/application/plot_offby_0_cancer.png}
    \caption{Distribution of unique phosphorylation sites identified by MaxQuant (MQ) and our  model(DNPS) for selected cancer-related proteins. The results demonstrate a significant expansion of the detectable phospho-landscape.}
    \label{fig:p_site_distribution}
\end{figure}

The statistical evaluation highlights a massive expansion of the p-site landscape. For the \textit{offby\_0} category, a total of 224,637 unique p-sites were identified across the dataset. While MaxQuant identified 40,598 sites, the \textit{de novo} model (DNPS) contributed 103,438 sites, with an additional 180,526 sites overlapping or being newly discovered in total (see Figure \ref{fig:p_site_distribution}). In the \textit{offby\_1\_snp} category, the expansion is even more pronounced, with DNPS identifying 77,078 unique sites compared to only 4,977 by MQ. It should be noted, however, that while these numbers are promising, they likely contain a higher noise floor, as stringent false discovery rate (FDR) filtering for \textit{de novo} PTMs is still an evolving area of research.

We took a quick look into the top 5 candidate cancer-related proteins frequently identified in our p-site pipeline:

\begin{itemize}
    \item \textbf{SRCIN1 (Q9C0H9):} This protein acts as a negative regulator of SRC kinase by activating CSK, thereby inhibiting downstream signaling pathways involved in cell migration \cite{UniprotQ9C0H9}. While databases like PhosphoSitePlus indicate potential regulation via phosphorylation, our model identified a high density of p-sites (e.g., over 80 sites in certain contexts), suggesting a complex regulatory "p-code" that warrants further biochemical validation.
    
    \item \textbf{NCOR1 (O75376):} A nuclear receptor corepressor that recruits histone deacetylases to mediate gene repression. Phosphorylation (e.g., via Akt) is known to regulate its dissociation from receptors like PPAR$\alpha$ \cite{NCOR1_PMC}. Our discovery of additional sites suggests a more nuanced control of metabolic gene activation than previously documented.
    
    \item \textbf{UBR4 (Q5T4S7):} An E3 ubiquitin-protein ligase involved in the N-degron pathway. Our data shows numerous previously uncharacterized p-sites alongside known ubiquitination sites, likely reflecting its role in orchestrating complex stress responses and protein turnover \cite{UniprotQ5T4S7}.
    
    \item \textbf{PML (P29590):} Crucial for the formation of PML-nuclear bodies (PML-NBs), this protein is a central hub for tumor suppression. PTMs are known to regulate its scaffolding function and antiviral responses \cite{PML_PMC}. The expanded p-site map provided by our model could clarify the dynamics of PML-NB assembly in cancer cells.
    
    \item \textbf{EGFR (P00533):} As a major therapeutic target in oncology, the phosphorylation of EGFR is well-studied, with over 100 known sites in specialized databases \cite{EGFR_PubChem}. Our model successfully recovered known regulatory tyrosines (e.g., Y1173) while proposing novel threonine and serine sites that may contribute to signaling crosstalk.
\end{itemize}

Despite these findings, the high number of predicted sites (particularly the 80 sites on SRCIN1) suggests that the current \textit{de novo} PTM output requires further structural filtering. The potential for false positives due to spectral noise or misassignment of mass shifts remains a challenge, necessitating more refined localized scoring in future iterations of the pipeline.

While the high density of predicted sites in proteins like SRCIN1 underscores the need for further structural filtering, our pipeline successfully identified several high-confidence regulatory markers with profound biological implications. The most compelling example is the detection of phosphorylation at \textbf{Serine 15 (p-Ser15)} on the liver isoform of glycogen phosphorylase, \textbf{PYGL} (P06737).

\begin{figure}[htbp]
    \centering
    \includegraphics[width=0.8\textwidth]{plots/application/PYGL_scores.png}
    \caption{Confidence distribution for the predictions validating p-Ser15 on PYGL. The identification is supported by 279 PSMs with a mean spectral angle (SA) of 0.7, indicating high reproducibility and spectral fidelity.}
    \label{fig:pygl_validation}
\end{figure}

The identification of p-Ser15 on PYGL is supported by substantial spectral evidence, comprising 279 Peptide-Spectrum Matches (PSMs) with a high mean spectral angle of 0.7 across all observations (see Figure \ref{fig:pygl_validation}). In liver metabolism, Ser15 acts as the central "on/off" switch for the enzyme; its phosphorylation by phosphorylase kinase (PHK) converts the inactive \textit{phosphorylase b} into the active \textit{phosphorylase a}, thereby driving the rate-limiting step of glycogenolysis—the breakdown of glycogen into glucose-1-phosphate \cite{Zois2022}.

Although PYGL is primarily known as the hepatic isoform, recent studies have highlighted its critical role in the "glycogen shunt" of cancer cells, particularly in glioblastoma (GBM). Under hypoxic conditions, tumor cells frequently undergo "isoform switching" or upregulate PYGL to utilize glycogen stores as a survival mechanism \cite{Favaro2012}. High expression of PYGL has been linked to poor prognosis in glioma patients and is strongly associated with hypoxia-inducible factor (HIF) signatures \cite{Zois2022, PURE_Ulster}. 

Our detection of p-Ser15-PYGL in these samples serves as a functional marker for the metabolic state of the tumor. The presence of the active \textit{phosphorylase a} form suggests that these cells are actively mobilizing glycogen to maintain energy homeostasis under metabolic stress. Furthermore, research in other cancer models (e.g., HCT116) has shown that p-Ser15 levels increase significantly under chemically induced hypoxia, paralleling other regulatory PTMs like O-GlcNAcylation \cite{PMC9240045}. The ability of our \textit{de novo} approach to confidently recover this specific regulatory site without prior database constraints demonstrates its potential to uncover metabolic drug targets and biomarkers that are central to cancer cell resilience.
