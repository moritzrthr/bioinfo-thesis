\chapter{Datasets}
\label{chap:datasets}

This chapter describes the data resources used for the fine-tuning and evaluation of the model. All spectral data were acquired using high-resolution mass spectrometry and represent a diverse range of peptide sequences and modifications.

\section{Fine-Tuning Data}

The fine-tuning process utilizes two distinct datasets to adapt the model to TMT-labeled spectra while maintaining performance on unlabeled data.

\subsection{TMT-labeled Dataset}

The primary dataset for adapting the model to isobaric labeling is derived from the PROSPECT-MultiPTM collection \cite{Zeng2024}. These data are based on the ProteomeTools project, a large-scale synthetic peptide library effort \cite{Zolg2017}.

\paragraph{Instrumentation and Fragmentation}
All spectra were acquired using Thermo Scientific Orbitrap instruments (Q Exactive and Orbitrap Fusion series). Fragmentation was performed exclusively using Higher-energy Collisional Dissociation (HCD), resulting in high-resolution MS2 spectra. The raw data are hosted via the PRIDE archive and ProteomeXchange \cite{Perez-Riverol2022}.

\paragraph{Labeling and Modifications}
The peptides in this dataset are labeled with Tandem Mass Tags (TMT). The chemical modification manifests as a specific mass shift at the peptide N-terminus and on the $\epsilon$-amino group of all Lysine (Lys) residues. The exact mass shifts follow the Unimod definitions and are encoded using the ProForma standard \cite{Leis2022}.

\paragraph{Dataset Statistics}
The TMT-labeled dataset is partitioned into training, validation, and test sets with the following dimensions:
\begin{itemize}
    \item \textbf{Training Set:} 3,683,888 Peptide-Spectrum Matches (PSMs) covering 75,268 unique peptides.
    \item \textbf{Validation Set:} 363,612 PSMs covering 7,751 unique peptides.
    \item \textbf{Test Set:} 364,867 PSMs covering 9,415 unique peptides.
\end{itemize}

\subsection{Non-TMT  Data (Replay Set)}

To prevent catastrophic forgetting during the fine-tuning process, a diverse reference dataset of unlabeled (non-TMT) spectra is included. This "Replay Set" consists of a mixture of 80\% MultiPTM data and 20\% data from the MassIVE Knowledge Base (MassIVE-KB) \cite{Wang2018}.

\paragraph{Technical Characteristics}
The instrumentation and fragmentation settings (Orbitrap HCD) are consistent with the TMT-labeled dataset to ensure technical compatibility. This set includes a wide variety of post-translational modifications and biological sequences.

\paragraph{Dataset Statistics}
The non-TMT reference data is partitioned into training, validation, and test sets with the following dimensions:
\begin{itemize}
    \item \textbf{Training Set:} 784,128 PSMs covering 289,568 unique peptides.
    \item \textbf{Validation Set:} 98,396 PSMs covering 23,004 unique peptides.
    \item \textbf{Test Set:} 93,453 PSMs covering 19,141 unique peptides.
\end{itemize}