\chapter{Datasets}
\label{chap:datasets}

Dieses Kapitel beschreibt die verwendeten Datensätze in der vorliegenden Arbeit.

\section{Überblick über die Datensätze}

Hier können Sie einen Überblick über die Datensätze geben, die in Ihrer Bachelor-Thesis verwendet werden. Zum Beispiel:

- Datensatz 1: Beschreibung, Quelle, Größe, etc.
- Datensatz 2: Beschreibung, Quelle, Größe, etc.

\section{Datenquellen}

Beschreiben Sie, woher die Daten stammen. Zum Beispiel:

- Öffentliche Datenbanken wie NCBI, Ensembl, etc.
- Eigene Experimente oder Simulationen.

\section{Datenverarbeitung}

Erklären Sie, wie die Daten vorverarbeitet wurden:

- Filterung, Normalisierung, etc.
- Tools oder Skripte, die verwendet wurden.

\section{Statistische Eigenschaften}

Fügen Sie Tabellen oder Abbildungen hinzu, die die Eigenschaften der Datensätze zeigen.

\begin{table}[h]
\centering
\caption{Übersicht der Datensätze}
\label{tab:datasets}
\begin{tabular}{|l|c|c|}
\hline
Datensatz & Anzahl Einträge & Quelle \\
\hline
Datensatz A & 1000 & NCBI \\
Datensatz B & 500 & Eigenes Experiment \\
\hline
\end{tabular}
\end{table}

Dies ist ein Mock-Beispiel. Passen Sie den Inhalt an Ihre tatsächlichen Datensätze an.