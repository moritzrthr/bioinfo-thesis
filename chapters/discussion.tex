\chapter{Discussion}
%and prospect

\section{Key Findings}

This work serves as a proof-of-concept that adapting transformer-based \textit{de novo} sequencing models to TMT-labeled proteomics data is not only feasible but yields significant biological insights. By accounting for the non-trivial systematic mass shifts and complex fragmentation patterns of Tandem Mass Tags, the model successfully recovered a vast majority of database-identified peptides while doubling the number of unique sequences identified in complex glioma samples. The discovery of 694 genomic-validated SNPs and the identification of the metabolic "on-switch" p-Ser15 on PYGL demonstrate that the model can reliably move "beyond the database search engine" to uncover regulatory and structural variants that are invisible to traditional methods.


\section{Limitations and Methodological Constraints}
\label{sec:limitations}

Despite the advancements in adapting \textit{de novo} architectures for TMT data, several limitations persist. A primary challenge encountered during the development was the relative sparsity of high quality TMT datasets with "good ground truth". The initial strategy utilized a "replay set" with the intention of fostering knowledge transfer between domains. However, later linear probing  revealed that the encoder  learns to distinguish between TMT-labeled and non-labeled spectra within the embedding space.

This distinction suggests that the decoder’s predictive capacity remains somewhat restricted to the specific modification patterns present in the TMT training data, potentially hindering true cross-domain transfer learning. To overcome this, a domain-agnostic preprocessing step is required to generate a universal embedding space. Such a "foundation model" approach would allow the decoder to apply knowledge learned from vast non-TMT datasets to the specific challenges of multiplexed proteomics, effectively combining the strengths of models like Casanovo \cite{Yilmaz2022} and specialized PTM-aware architectures.

Additionally future refinement of the finetuning dataset could benefit from including unmodified TMT data to provide a more balanced representation of the peptide space. Finally, extending the model architecture with specialized "heads" for downstream tasks—such as PTM localization scoring or direct PTM classification—would further enhance the interpretability and confidence of the sequencing results \cite{Zhu2017} or could be one task for itself.




\subsection{Future Improvements and Clinical Utility}
The application of the developed model to complex datasets, such as those derived from glioma research, represents only the initial stage of its potential utility. While current collaborations provided a foundation, the depth of biological discovery can be significantly enhanced. Beyond identifying single amino acid variants, future iterations should integrate proteogenomic workflows. By incorporating six-frame translations or patient-specific transcriptomes, thousands of currently "high-confidence" but unassigned spectra could be systematically validated \cite{Nesvizhskii2014}. This approach would facilitate the discovery of cryptic peptides, alternative splicing events, and non-canonical open reading frames (ORFs) that remain invisible to standard workflows.

Furthermore, the synergy between \textit{de novo} sequencing and TMT multiplexing offers a scalable framework for precision medicine. By directly linking discovered variants to patient-specific MS3 channels, tumor heterogeneity can be characterized with unprecedented throughput. Future developments could focus on a real-time sequencing pipeline integrated with clinical metadata, transitioning mass spectrometry from a retrospective tool into a proactive platform for personalized oncology.


