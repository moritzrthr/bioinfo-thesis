\chapter{Methods}





\label{ch:methods}







\section{Fine-tuning Strategy and Model Adaptations}

The approach builds upon the Modanovo framework, a transformer-based architecture designed for the identification of post-translational modifications (PTMs) using experimental spectra \cite{KlaprothAndrade2025}.


The transition from unlabeled or label-free spectra to TMT-multiplexed data requires specific adaptations of the underlying deep learning model. In this work, this specialized version is introduced as \textbf{Modanovo-TMT}. The fine-tuning process involves adjusting the model to recognize TMT labels not as global experimental parameters, but as specific chemical modifications integrated into the sequencing vocabulary. 


\subsection{Tokenization and Vocabulary Expansion}

To accommodate TMT labeling, the model's tokenization strategy was expanded. Modanovo utilizes a residue-based vocabulary where each token represents either a standard amino acid or a specific amino acid-PTM combination \cite{KlaprothAndrade2025}. For this study, the configuration was adjusted to include TMT-specific tokens. These tokens account for the fixed mass shifts on N-termini and Lysine (K) residues.



Specifically, the vocabulary was extended by the following residues and their corresponding mass shifts:

\begin{itemize}

\item \textbf{K[+229.163]}: Lysine with TMT10/16 label.

\item \textbf{[+229.163]-}: TMT10/16 label at the peptide N-terminus.

\item \textbf{K[+343.206]}: Lysine with both TMT and GlyGly (ubiquitination) modification.

\item \textbf{K[+271.173]}: Lysine with both TMT and Acetyl modification.

\item \textbf{K[+243.179]}: Lysine with both TMT and Methyl modification.
\end{itemize}

Following the Modanovo initialization protocol, the embeddings for these new tokens were initialized by averaging the embeddings of their constituent components (e.g., the base amino acid embedding and the modification-specific shift) to leverage pre-learned chemical representations \cite{KlaprothAndrade2025}.

\subsection{TMT Covariate Embedding}

To enable the decoder to account for systematic shifts in fragmentation patterns and physicochemical properties induced by TMT labeling, we introduce a categorical conditioning mechanism. This allows the model to explicitly distinguish between TMT-labeled and unlabeled spectra at a global level.

In the baseline transformer architecture (e.g., Casanovo [Yilmaz2022]), the precursor embedding ($\mathbf{prec_emb}$) encodes the precursor ion's mass-to-charge ratio ($m/z$) and charge state ($z$) using sinusoidal positional encodings passed through dedicated linear projection layers. 

Analogous to the embedding of precursor features (precursor mass and charge), we define a
learnable TMT-specific embedding. For each spectrum, a binary indicator
\(
f_{\mathrm{TMT}} \in \{0,1\}
\)
encodes the presence or absence of TMT labeling and is mapped through an embedding layer:

\[
\mathbf{E}_{\mathrm{TMT}} = \mathrm{Embedding}(f_{\mathrm{TMT}}) \in \mathbb{R}^{d_{\mathrm{model}}}.
\]

The resulting vector is integrated into the latent representation via additive fusion.
Specifically, it is added to the precursor embedding prior to decoding:

\[
\mathbf{prec\_emb}_{\mathrm{conditioned}}
= \mathbf{prec\_emb} + \mathbf{E}_{\mathrm{TMT}}.
\]

By injecting this information at the level of the precursor representation—effectively
seeding the start of the decoding process—the transformer can adapt its internal
representations to the chemical environment associated with TMT-labeled peptides.
This conditioning strategy is computationally efficient, as it preserves the model
dimensionality while providing a strong global signal that guides de novo sequencing
depending on the labeling state of the sample. The adapted architecture is illustrated in Figure \ref{fig:architecture}.



\begin{figure}[H]
    \centering
    %\includegraphics[width=\textwidth]{architecture_diagram.pdf}
    \includegraphics{plots/methods/architecture_tmt.png}
    \caption{
    Schematic representation of the adapted transformer architecture based on Casanovo. Allowing the identification of TMT-labeled post-translationally modified peptides directly from tandem mass (MS2) spectra. The model is trained starting with weight initialization from Modanovo’s pre-trained weights. The model components for the amino acid (AA) embeddings and final linear layer are expanded to allow the TMT-tokens. The TMT status is fed as a covariate embedding into the decoder alongside the precursor and previous predicted residues.
    }
    \label{fig:architecture}
\end{figure}

\section{Finetuning Datasets}

\subsection{Data Selection and Composition}
For the training and evaluation of Modanovo-TMT, a robust dataset was curated to ensure high-quality spectral representations. The fundamental requirement for supervised learning in this context is the availability of database searching derived sequences associated with high-resolution fragment spectra. Data were integrated from various sources, initially stored in CSV and mzML formats, and subsequently compiled into a unified Mascot Generic Format (MGF) file. This format allows for a streamlined input pipeline where the peptide sequence is explicitly linked to its corresponding spectrum \cite{Deutsch2012}.


The fine-tuning process utilizes two distinct datasets to adapt the model to TMT-labeled spectra while maintaining performance on unlabeled data (see Figure \ref{fig:datasplits}).

\begin{figure}[H]
    \centering
    % width=\textwidth sorgt dafür, dass das Bild exakt so breit wie der Text ist.
    % Mit 0.8\textwidth kannst du es etwas kleiner machen, falls es zu wuchtig wirkt.
    \includegraphics[width=0.9\textwidth]{plots/methods/datasplits.png}
    
    \caption{Composition of the fine-tuning dataset. The training dataset is dominated by TMT-labeled multi-PTM peptide-spectrum matches (82\%), supplemented by a 18\% Replay Set to preserve model robustness and recall on unlabeled spectra.}
    \label{fig:datasplits}
\end{figure}

\subsection{TMT-labeled Data}
The primary dataset for adapting the model to isobaric labeling is derived from the PROSPECT-MultiPTM collection \cite{Zeng2024}. These data are based on the ProteomeTools project, a large-scale synthetic peptide library effort \cite{Zolg2017}.

\paragraph{Instrumentation and Fragmentation}
All spectra were acquired using Thermo Scientific Orbitrap instruments (Q Exactive and Orbitrap Fusion series). Fragmentation was performed exclusively using Higher-energy Collisional Dissociation (HCD), resulting in high-resolution MS2 spectra.

\paragraph{Labeling and Modifications}
The peptides in this dataset are labeled with Tandem Mass Tags (TMT). The chemical modification manifests as a specific mass shift at the peptide N-terminus and on the $\epsilon$-amino group of all Lysine (K) residues. The exact mass shifts follow the Unimod definitions and are encoded using the ProForma standard \cite{Leis2022}.

\paragraph{Dataset Statistics}
The TMT-labeled dataset is partitioned as follows:
\begin{itemize}
    \item \textbf{Training Set:} 3,683,888 PSMs covering 75,268 unique peptides.
    \item \textbf{Validation Set:} 363,612 PSMs covering 7,751 unique peptides.
    \item \textbf{Test Set:} 364,867 PSMs covering 9,415 unique peptides.
\end{itemize}

\subsection{Non-TMT Data (Replay Set)}
To prevent catastrophic forgetting during the fine-tuning process, a diverse reference dataset of unlabeled (non-TMT) spectra is included. This "Replay Set" consists of a mixture of 80\% MultiPTM data and 20\% data from the MassIVE Knowledge Base (MassIVE-KB) \cite{Wang2018}.

\paragraph{Dataset Statistics}
The non-TMT reference data is partitioned as follows:
\begin{itemize}
    \item \textbf{Training Set:} 784,128 PSMs covering 289,568 unique peptides.
    \item \textbf{Validation Set:} 98,396 PSMs covering 23,004 unique peptides.
    \item \textbf{Test Set:} 93,453 PSMs covering 19,141 unique peptides.
\end{itemize}

\subsection{Quality Filtering and Pre-processing}
To minimize noise and prevent the model from learning experimental artifacts, several quality filtering steps were applied:
\begin{itemize}
    \item \textbf{Peak Cleaning:} Spectra containing no intensity information or empty peaks were removed.
    \item \textbf{Bias Mitigation:} To prevent overfitting to hyper-abundant peptides, a threshold of 229 PSMs per unique peptide sequence was enforced.
    \item \textbf{Data Leakage Prevention:} Following the \textit{Modanovo} protocol, a strict data split was implemented. Peptides with specific modifications (e.g., $PEP[ph]$) were assigned to the same split as their unmodified counterparts ($PEP$) to ensure the model learns chemical principles rather than memorizing sequences \cite{KlaprothAndrade2025}.
\end{itemize}



The final dataset was structured into an 80/10/10 split (training, validation, and testing). To specifically address the TMT expansion, an 80/20 balance between TMT-labeled and unlabeled spectra was maintained, and all non-TMT spectra originating from TMT-specific experiments were removed to ensure label consistency. Furthermore, Unimod syntax was translated into mass-shift syntax (e.g., [+229.163]) to align with the model's vocabulary.


\section{Training Protocol}

Model fine-tuning was initialized from the publicly available Modanovo checkpoint, allowing the model to build upon previously learned representations of peptide fragmentation and spectral structure \cite{KlaprothAndrade2025}. In contrast to partial adaptation strategies, all model parameters were updated during fine-tuning, i.e.\ no layers were frozen, enabling global adaptation to TMT-specific fragmentation effects and modification patterns.

The underlying transformer architecture was kept identical to the base Modanovo configuration, comprising a model dimension of $d_{\text{model}} = 512$, 8 self-attention heads, a feed-forward dimension of 1024, and 9 layers each in the encoder and decoder stacks. This architectural consistency ensures that any observed performance differences can be attributed to the fine-tuning procedure rather than structural changes.

Optimization hyperparameters were deliberately chosen to favor stable adaptation of the pre-trained weights. A low learning rate of $1 \times 10^{-6}$ was used to prevent catastrophic forgetting while still permitting gradual adjustment to TMT-induced shifts in fragmentation behavior. To further stabilize early training dynamics, a warm-up phase of two epochs was applied. Regularization was introduced via a weight decay of $1 \times 10^{-5}$ and label smoothing with a factor of 0.01, improving generalization in the presence of heterogeneous modification patterns.

Training was performed using mixed-precision arithmetic with \textit{bf16} precision, reducing memory consumption and improving computational efficiency on modern GPU architectures without compromising numerical stability \cite{Micikevicius2017}. Model selection was based on validation loss, and the checkpoint with the lowest validation loss was retained for all downstream analyses.


\section{Evaluation Strategy and Performance Metrics}


To rigorously assess the performance of Modanovo-TMT, a multi-faceted evaluation framework was established. The primary objective is to determine how well the model generalizes to TMT-labeled spectra and various post-translational modifications (PTMs) compared to traditional database-driven assignments.

\subsection{Confidence Scoring and Peptide Ranking}
Each peptide-spectrum match (PSM) generated by the model is assigned a confidence score to facilitate ranking and quality control. Following the architecture of Transformer-based models like Casanovo and Modanovo, we derive a peptide-level score by calculating the arithmetic mean of the individual amino acid confidence scores, which are obtained from the softmax output at each decoding step \cite{Yilmaz2022}.

To ensure the  plausibility of the predictions, a mass-matching constraint is applied. If the calculated mass of the predicted sequence (including PTMs and TMT labels) deviates from the observed precursor mass beyond a defined tolerance (e.g., 10 ppm), the peptide score is penalized. This integration of spectral evidence and thermodynamic constraints is crucial for distinguishing between high-confidence sequences and plausible but incorrect mass-shift combinations.

\subsection{Precision-Coverage Analysis and Stratification}
The core metric for evaluating the model's predictive power is the precision-coverage curve. This allows for a threshold-independent assessment of how many peptides can be identified at a given reliability level. For this study, we specifically focus on the Area Under the Precision-Coverage Curve (AUPCC), calculated using the trapezoidal rule \cite{Pedregosa2011}.

A critical aspect of our evaluation is the **stratified analysis**. To understand the specific impact of the TMT expansion, we evaluate the performance separately for:
\begin{itemize}
    \item \textbf{TMT-labeled spectra:} To measure the success of the model adaptation to systematic mass shifts.
    \item \textbf{Unlabeled (non-TMT) spectra:} To ensure that the model retains its general sequencing capabilities without losing performance on standard data (preventing "catastrophic forgetting").
\end{itemize}

Precision ($P$) and Coverage ($C$) at a score threshold $t$ are defined as:
\begin{equation}
    P(t) = \frac{|\text{Correct PSMs with score} \geq t|}{|\text{Total predictions with score} \geq t|}
\end{equation}
\begin{equation}
    C(t) = \frac{|\text{Predictions with score} \geq t|}{|\text{Total ground truth identifications}|}
\end{equation}

A PSM is considered correct if the sequence exactly matches the ground truth identified by a database search (e.g., MaxQuant or MSFragger), treating isobaric amino acids such as Leucine, Isoleucine and PyroGlu-Q, PyroGlu-E as equivalent \cite{KlaprothAndrade2025}.
We evaluate the precision for specific PTM-amino acid combinations. For a given modification (e.g., Phosphorylation at T), we subset the ground truth data to include all peptides containing this specific shift. This granular view ensures that the model's ability to handle complex, multiplexed PTM patterns is validated across both TMT and non-TMT backgrounds.


The evaluation follows a peptide-centric approach: a single ground truth peptide contributes to the curves of all modifications it contains. For instance, a peptide sequence such as \enquote{[+229.997]-PEPT[+79.966]IDEK[+14.016]} is included in both the precision-coverage curve for phosphorylated threonine (T[+79.966]) and monomethylated lysine (K[+14.016]). In all subsequent analyses, the performance on unmodified peptides (light grey in Figure \ref{fig:PCfacets}) serves as a reference baseline.




\section{Application to Clinical Glioma Data}

To evaluate the practical utility and robustness of Modanovo-TMT in a real-world clinical context, we applied it to a large-scale, independent glioma dataset. Unlike the synthetic and curated libraries used for fine-tuning, this dataset represents the inherent complexity of clinical proteomics, characterized by high dynamic range and heterogeneous post-translational modifications.


\subsection{Experimental Dataset: The ClinSpect-M Glioma Cohort}

Modanovo-TMT was applied on an unpublished clinical dataset provided by the Kusterlab \cite{KusterLab2026}. This cohort comprises approximately 300 Glioma patient samples, representing a diverse spectrum of tumor grades and molecular subtypes like Astrocytoma, Oligodendroglioma, and Glioblastoma. 

\subsection{Data Acquisition and Large-Scale Preprocessing}

The phospho-enriched dataset consists of approximately 6.5 million tandem mass spectra, originally acquired as Orbitrap-based raw files (.raw). To ensure compatibility with the model input, raw files were converted into Mascot Generic Format (MGF) using the \texttt{ThermoRawFileParser} (v2.0.0) \cite{Hulstaert2020}. 

During preprocessing, all MS1 precursor scans and MS3 reporter ion scans were excluded from the sequencing workflow. The focus was restricted to MS2 fragment spectra, which contain the peptide backbone information necessary for sequence reconstruction. 

\subsection{Peptide Identification Pipeline}

To achieve a comprehensive analysis of the Glioma proteome, we employed a dual-strategy approach: \textit{de novo} sequencing for discovery and a database-driven search as a validation baseline.

\paragraph{De Novo Sequencing Configuration}
The sequencing of the TMT-labeled spectra was performed using the adapted Modanovo framework. To ensure high-quality sequence predictions and to accommodate the systematic shifts introduced by TMT, the following parameters were applied:
\begin{itemize}
    \item \textbf{Mass Tolerances:} The precursor mass tolerance was set to 50 ppm to account for potential drift in large-scale clinical datasets. The isotope error range was restricted to $[0, 3]$.
    \item \textbf{Sequence Constraints:} A minimum peptide length of 6 amino acids and a maximum of 100 were enforced. For the decoding process, a beam search with a width of $n\_beams = 1$ was utilized, focusing on the top-ranked match to maximize throughput.
   
\end{itemize}

\paragraph{Database Search Configuration (MaxQuant Baseline)}
To provide a complementary ground-truth baseline and validate the \textit{de novo} results, all Glioma datasets were processed using MaxQuant (version 2.1.3.0) \cite{Tyanova2016}. This step was done by our collaborators and we obtained the results from them. This step is crucial to determine the "searchable" fraction of the proteome and to identify which spectra remain "unexplained" by traditional methods, thereby highlighting the discovery potential of Modanovo-TMT beyond database-driven searches.

The search was conducted against the human reference proteome (UniProt UP000005640) with parameters harmonized to the experimental design:
\begin{itemize}
    \item \textbf{Protease:} Trypsin/P was specified, allowing for cleavage C-terminal to Lysine and Arginine, even when followed by Proline.
    \item \textbf{Fixed Modifications:} Carbamidomethylation of cysteine (+57.021 Da) and TMT labeling of Lysine and the peptide N-terminus (+229.163 Da) were set as static modifications.
    \item \textbf{Variable Modifications:} Oxidation (M), acetylation (n-Term) and phosphorylation (S/T/Y) were included in the search space.
\end{itemize}





\subsection{Peptide Alignment and Sequence Validation}
To validate the biological origin of the predicted \textit{de novo} sequences and to distinguish between known peptides and potential novel discoveries, a sequence alignment against the reference proteome was performed. This step is crucial because \textit{de novo} models predict sequences solely based on spectral features, which may include errors or biologically plausible variations not present in the reference database.

\paragraph{Mass-Spectrometric Ambiguities and Encoding Strategy}
A fundamental challenge in de novo peptide sequencing is the existence of isobaric or near-isobaric amino acid residues. 

Since deep learning models primarily operate on mass-to-charge ratios ($m/z$) and the resulting mass differences, they essentially identify mass shifts rather than unique chemical identities.
While transformer-based architectures can theoretically also learn to consider aminoacid context when mass shifts overlap, this still remains unevaluated and therefore we resolve the ambiguities. 

We identified several critical ambiguities:
\begin{itemize}
    \item \textbf{I/L Equivalence:} Leucine (L) and Isoleucine (I) are structural isomers with an identical monoisotopic mass of $113.08406$ Da, making them indistinguishable in HCD spectra.
    \item \textbf{Deamidation-induced Ambiguities:} The deamidation of Glutamine (Q) and Asparagine (N) results in a mass increase of $+0.984016$ Da. This shift leads to these pairs:
    \begin{itemize}
        \item Glutamic acid (E, $129.042593$ Da) and deamidated Glutamine (Q[+0.98], $129.042594$ Da).
        \item Aspartic acid (D, $115.026943$ Da) and deamidated Asparagine (N[+0.98], $115.026943$ Da).
    \end{itemize}
    \item \textbf{Pyro-glutamate Formations:} Ambiguities also occur between Pyro-glu E and Pyro-glu Q. However, as these appeared in less than 1\% of the detected spectra in our dataset, they were not explicitly encoded to avoid over-complicating the search space.
\end{itemize}

\paragraph{BLASTp Configuration and Ambiguity Handling}
Alignment was executed using \texttt{Protein-Protein BLAST} (version 2.17.0+) against the human reference proteome (UniProt UP000005640). Because the reference proteome consists of unmodified sequences, a direct match of a deamidated peptide (predicted as D or E by the model) against a genomic N or Q would fail under standard parameters.

To resolve this, we implemented a specialized encoding strategy using IUPAC ambiguity codes:
\begin{itemize}
    \item All predicted \textbf{D} residues (which could be $N[+0.98]$) were replaced with \textbf{B} (asparagine or aspartic acid).
    \item All predicted \textbf{E} residues (which could be $Q[+0.98]$) were replaced with \textbf{Z} (glutamine or glutamic acid).
\end{itemize}

An alternative approach would have been the implementation of a custom substitution matrix. However, the BLASTp  does not support user-defined matrices for short peptide searches without extensive modification of the source code. Therefore, we utilized the \texttt{PAM30} matrix, which is optimized for short sequences and natively supports the \texttt{B} and \texttt{Z} codes.

% --- Visual Example for Thesis ---
\vspace{1em}
\noindent \textbf{Example of IUPAC Encoding for Deamidation Alignment:}
\begin{center}
\small
\begin{tabular}{rll}
    \textbf{Reference Proteome:} & \texttt{A V G \textbf{D} L T S \textbf{Q} R} & \\
    \textbf{De-novo Prediction:} & \texttt{A V G \color{red}{N} L T S \color{red}{E} R} & \textit{\footnotesize (Potential Mismatches)} \\
    \noalign{\smallskip}
    \cline{1-2}
    \noalign{\smallskip}
    \textbf{Standard BLASTp:}   & \texttt{A V G \color{red}{-} L T S \color{red}{-} R} & \textbf{\color{red}{Mismatch}} \\
    \textbf{Our Strategy (PAM30):} & \texttt{A V G \color{blue}{B} L T S \color{blue}{Z} R} & \textbf{\color{blue}{Perfect Match*}} \\
\end{tabular}
\end{center}
\textit{\footnotesize *Note: Both sequences are converted to IUPAC codes B (N/D) and Z (Q/E) prior to alignment, allowing the PAM30 matrix to score them as identical residues.}
\vspace{1em}

\paragraph{Alignment Parameters}
The alignment parameters were adjusted to account for the short nature of tryptic peptides. We set an E-value threshold of $2000$, as the small search space of a single peptide often results in high E-values despite perfect sequence identity. 
Furthermore, we applied a query coverage threshold (\texttt{qcov\_hsp\_perc}) of 80\%. This constraint ensures that at least 80\% of the \textit{de novo} predicted sequence aligns with the database entry, thereby filtering out spurious, short-match alignments.

Another adjustment was the deactivation of composition-based statistics (\texttt{comp\_based\_stats}). In default settings, many alignment tools (such as BLAST) adjust the scoring matrices based on the amino acid composition of the sequences being compared to account for biased distributions \cite{Altschul2001}. However, in the context of \textit{de novo} sequencing, where sequences are often short, these statistics can lead to score inflation or, conversely, penalize short but biologically valid matches \cite{Frank2005}. 

\paragraph{Post-Alignment Filtering and SNP Analysis}
The resulting alignments were further enriched to identify Single Nucleotide Polymorphisms (SNPs) and truncation events. To account for sequences extending beyond protein termini or alignment gaps, the full query and target sequences were retrieved. 
The number of mismatches was calculated by considering the \texttt{B} and \texttt{Z} equivalences. For each remaining mismatch, an automated codon-lookup was performed. A mismatch was classified as "SNP-explainable" if the transition between the predicted amino acid and the reference residue could be achieved by a single nucleotide substitution in the underlying codon. 

Cases involving gaps or truncations where the query extended beyond the reference boundaries were excluded from the SNP analysis to maintain high confidence in the mutation mapping.

\subsection{Genomic Validation and Integration of Panel Sequencing Data}

To substantiate the biological relevance of the \textit{de novo} predicted sequences, particularly those harboring potential amino acid substitutions, we integrated patient-specific genomic evidence. For the ClinSpect-M cohort, genomic panel sequencing data (covering approximately 500 cancer-relevant genes) was available.

\paragraph{Processing of Genomic Variants}
The genomic data, provided in annotated JSON format, was processed using the Nirvana clinical variant annotator (v3.2.3) based on the GRCh37 (hg19) genome assembly. To identify potential tumor-specific peptides and single amino acid variants (SAAVs), somatic mutations were filtered using the Variant Effect Predictor (VEP) \cite{McLaren2016}. 

\paragraph{Mapping and Evidence Filtering}
For each \textit{de novo} peptide that aligned to a protein originating from the 500-gene panel with only a few mismatches, we performed a precise positional mapping. The Uniprot identifiers from the proteomic alignment were mapped to Ensembl transcript IDs used in the VEP output. 
A peptide candidate was considered "genomically validated" if:
\begin{enumerate}
    \item The predicted mismatch relative to the UniProt reference sequence occurred at the exact position of a genomic variant identified in the panel sequencing.
    \item The amino acid substitution predicted by the \textit{de novo} model matched the transcript-level prediction (HGVSp) from the genomic data.
\end{enumerate}
This integrated workflow allows for the identification of non-canonical peptides that are absent from standard reference databases but supported by the patient's individual mutational landscape.



\section{Identification and Validation of Phosphorylation Sites}

Beyond sequence variations, the precise localization of post-translational modifications (PTMs), particularly phosphorylations, is critical for understanding cellular signaling. We implemented a standardized workflow to map and validate predicted modification sites.

The localization of phosphorylation sites (p-sites) was performed by integrating \textit{de novo} predictions from Modanovo-TMT and traditional database-driven results from MaxQuant. 

For each predicted phosphopeptide, the unmodified sequence was first aligned to the reference proteome as described in the previous section. By utilizing the precise start and end coordinates of the alignment, we translated the local modification index (the position within the peptide) into a global protein position. This allows for a direct comparison of modification sites across different peptides and experiments, even if the underlying peptide sequences differ due to alternative cleavage or truncations.

To ensure the reliability of the identified p-sites, we had a look at the evidence supporting each site taking into account the following factors:
    
\begin{itemize}
    \item \textbf{Site Redundancy:} Number of supporting Peptide-Spectrum Matches (PSMs) and overlapping sequences from alternative cleavage events.
    \item \textbf{Quality Metrics:} \textit{De novo} confidence scores for the PSMs and the alignment query coverage.
    \item \textbf{Cross-Platform Consistency:} Identification of "consensus sites" predicted by both MaxQuant and \textit{ModanovoTMT}.
    \item \textbf{Literature Comparison:} Cross-referencing mapped positions with previous studies.
\end{itemize}

\section{Spectral Evidence and Intensity Prediction}

While sequence alignment and genomic mapping provide biological context, the physical validity of a \textit{de novo} prediction must be confirmed by comparing the experimental MS/MS spectrum with the expected fragmentation pattern of the predicted sequence.

\subsection{Spectral Angle (SA) as a Quality Metric}
To quantify the similarity between the experimental spectrum and a theoretical prediction, we utilize the Spectral Angle ($SA$). Unlike the traditional Pearson correlation, the $SA$ is less sensitive to high-intensity peaks and provides a more robust measure of relative fragment ion intensities \cite{Toprak2014}. The $SA$ is defined as:

\begin{equation}
    SA = 1 - \frac{2 \cdot \arccos\left( \frac{\mathbf{u} \cdot \mathbf{v}}{||\mathbf{u}|| \cdot ||\mathbf{v}||} \right)}{\pi}
\end{equation}

where $\mathbf{u}$ represents the vector of intensities from the experimental spectrum and $\mathbf{v}$ the predicted intensities. A value of $1$ indicates a perfect match, while $0$ indicates no similarity.

\subsection{Intensity Prediction via Koina and Prosit}
Since TMT labeling and various PTMs notably alter fragmentation energy and peak intensities, we utilized the \texttt{Prosit\_2024\_intensity\_PTMs\_gl} model, accessed via the Koina federated prediction service \cite{Gessulat2019, Lautenbacher2024, Gabriel2025}. This deep learning model is specifically trained to handle TMT-labeled peptides and a wide array of PTMs.

\paragraph{Implementation Workflow}
For each high-scoring \textit{de novo} peptide, the predicted sequence (including modifications) was converted into the standardized Unimod format. These strings were sent to the Koina API along with the experimental parameters, specifically setting the Collision Energy (CE) to 32 to match the Orbitrap acquisition settings. The resulting theoretical intensities for b- and y-ions were then used to calculate the spectral angle $SA$. This step ensures that our discoveries are not only biologically plausible but also physically consistent with the raw mass spectrometric data.
