\chapter{Introduction}
\label{chap:introduction}

\section{Proteins and PTMs in Cancer}
Proteins are the primary functional units of the cell, executing the vast majority of biological processes. While the genome provides the blueprint, the proteome reflects the actual physiological state of an organism. In the context of complex diseases such as cancer, the importance of studying proteins becomes even more critical. Malignant transformation is often driven not just by the presence of certain proteins, but by their dynamic regulation through post-translational modifications (PTMs) \cite{Mertins2016}.

PTMs, such as phosphorylation, acetylation, and methylation, act as molecular switches that control protein activity, localization, and interaction networks. In neuro-oncology, particularly in the study of aggressive brain tumors like glioma, aberrant phosphorylation patterns are known to drive oncogenic signaling pathways, contributing to tumor growth and therapy resistance \cite{Jiang2025}. Identifying these modified proteins is essential for understanding the molecular landscape of a patient's tumor and for the development of targeted therapies. However, a vast portion of these critical biological signals remains hidden from researchers—a phenomenon often described as the "dark proteome."


\section{Mass Spectrometry: State of the Art to Analyze the Proteome}

To decipher the complexity of the proteome, Mass Spectrometry (MS)-based proteomics has emerged as the gold-standard technology. At its core, a mass spectrometer acts as an extremely precise molecular scale. In a typical "bottom-up" workflow, proteins are extracted from biological samples and digested into smaller fragments called peptides \cite{Aebersold2016}. These peptides are easier to measure and serve as proxies for the original proteins.


The identification of these peptides occurs via Tandem Mass Spectrometry (MS/MS). In this process, the instrument first weighs the intact peptide and then energetically breaks it into smaller fragment ions. The resulting MS/MS spectrum is a "fingerprint" showing the masses of these fragments. By analyzing the gaps between these mass peaks, which correspond to specific amino acids, the peptide's sequence can be reconstructed \cite{Steen2004}.

To increase efficiency in clinical or large-scale research, multiple samples are often labeled with Tandem Mass Tags (TMT). This chemical labeling technology allows for the simultaneous analysis (multiplexing) of up to 18 samples in a single experiment, enabling a direct comparison of protein levels across different conditions or patients \cite{Thompson2003}. 


\section{Database-Driven Searches}
The standard way of interpreting the spectrum (output of the mass spectrometer) is using a database search engine. This method compares experimental spectra against a predefined spectral library of known peptide sequences. While effective for "standard" proteins, database search engines struggle with PTMs. To find a modified peptide, the search engine must be told exactly which modification to look for. Including many possible PTMs leads to a "combinatorial explosion" that exponentially increases computation time and leads to higher false-discovery rates \cite{Chick2015}. Consequently, many spectra originating from unexpected PTMs or single amino acid variants (SAVs) in cancer cells are simply discarded as "unidentified."

\textit{De novo} peptide sequencing offers a complementary solution by predicting sequences directly from the spectra without needing a database. While recent deep learning models like \textit{Modanovo} have revolutionized this field \cite{KlaprothAndrade2025}, they face a hurdle: they were not designed for the systematic mass shifts and altered fragmentation patterns introduced by TMT labeling. Since TMT is often used for high-quality clinical data, this creates a gap: we have the data to find new cancer insights, but current de novo peptide sequencing tools cannot handle it yet.

\section{Objectives of this Thesis}
The primary objective of this thesis is not merely to expand a model, but to unlock the biological potential of TMT-labeled clinical datasets through adapted \textit{de novo} sequencing. By bridging the gap between deep learning and multiplexed proteomics, we aim to uncover biological insights in glioma data that remain invisible to standard workflows.

The specific goals of this thesis are:
\begin{enumerate}
    \item \textbf{Model Adaptation and Data Curation:} To adapt the \textit{Modanovo} architecture to handle the specific chemical signatures of TMT labeling. A part of this goal is the curation of high-quality TMT-labeled datasets to serve as  a training foundation.
    \item \textbf{Uncovering the Dark Proteome in Glioma:} To apply the adapted model to actual patient data from glioma studies. We aim to identify novel PTMs and SAVs that have been missed by database searches, thereby providing a more comprehensive molecular characterization of the tumor.
    \item \textbf{Clinical and Methodological Impact:} To evaluate the potential of this approach to improve clinical proteogenomics, providing a pathway for more sensitive biomarker discovery and a better understanding of the regulatory networks in cancer.
\end{enumerate}

Ultimately, this thesis explores the potential of de novo sequencing to transcend the limitations of predefined databases, enabling a more comprehensive view of the proteome in multiplexed quantitative studies.