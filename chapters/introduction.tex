\chapter{Introduction}
\label{chap:introduction}



\section{The Challenge of the "Dark Proteome" and PTMs}
Mass spectrometry (MS)-based proteomics has revolutionized our understanding of cellular processes by allowing the large-scale identification and quantification of proteins. However, a significant portion of the acquired MS/MS spectra—often referred to as the "dark proteome"—remains unassigned by traditional database-driven identification strategies (DBIS) \cite{Aebersold2016}. One major reason for this gap is the immense complexity of post-translational modifications (PTMs). PTMs, such as phosphorylation, ubiquitination, and methylation, are crucial regulators of protein function, localization, and signaling pathways, especially in diseases like cancer \cite{Mertins2016}.

Traditional search engines like MaxQuant or Sequest rely on predefined databases. Including a wide array of PTMs in these searches leads to a combinatorial explosion of the search space, which not only increases computational time but also drastically reduces statistical power and increases the false discovery rate (FDR) \cite{Chick2015}. Consequently, DBIS are often "blind" to unexpected or multiple PTMs occurring on the same peptide, leaving biologically critical information hidden in the unassigned data.

\section{De Novo Sequencing and the TMT Gap}
\textit{De novo} peptide sequencing offers a powerful alternative by predicting the amino acid sequence directly from the fragment spectra without a reference database. Recent advances in deep learning, particularly transformer-based models like \textit{Casanovo} and its PTM-specialized derivative \textit{Modanovo}, have pushed the boundaries of accuracy in this field \cite{Klaproth2024}. These models have shown great potential in uncovering novel PTM patterns and even single nucleotide polymorphisms (SNPs) in highly mutated samples, such as those found in glioma research \cite{Smith2019}.

Despite these advances, a significant barrier remains: Tandem Mass Tag (TMT) labeling. TMT is the gold standard for multiplexed quantitative proteomics, enabling the simultaneous analysis of multiple clinical samples. However, TMT labeling introduces systematic mass shifts and significantly alters the fragmentation patterns (e.g., favoring b-ions over y-ions) \cite{Shen2018}. Current \textit{de novo} models, including \textit{Modanovo}, were primarily trained on label-free data and therefore fail to accurately interpret TMT-labeled spectra. This creates a critical bottleneck for clinical proteogenomics, where TMT is the preferred workflow.

\section{Objectives of this Thesis}
The objective of this thesis is to bridge the gap between high-performance \textit{de novo} sequencing and TMT-based quantitative proteomics. We propose an adaptation of the \textit{Modanovo} framework specifically designed to handle the chemical signatures of TMT labeling. The focus lies on:
\begin{itemize}
    \item \textbf{Fine-Tuning:} Adapting the model weights using a large-scale, TMT-labeled dataset to learn specific mass shifts and fragmentation biases.
    \item \textbf{Architectural Enhancements:} Expanding the token vocabulary to include TMT-specific modifications.
    \item \textbf{Conditioning:} Implementing covariate embeddings to explicitly inform the model about the TMT status of a spectrum.
    \item \textbf{Evaluation:} Validating the model on a complex glioma dataset to demonstrate its ability to recover PTMs and SNPs where standard DBIS like MaxQuant reach their limits.
\end{itemize}
By unlocking the sequence information in TMT-labeled "dark" spectra, this work aims to provide deeper biological insights into the regulatory networks of cancer.