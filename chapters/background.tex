\chapter{Background}
\label{ch:background}

In this chapter, the fundamental principles of mass spectrometry-based proteomics and the computational strategies for peptide identification are discussed. Particular focus is placed on the challenges introduced by chemical labeling and the emergence of deep learning models in \textit{de novo} sequencing.

\section{Mass Spectrometry-Based Proteomics}

\subsection{Bottom-Up Proteomics Workflow}
Mass spectrometry (MS)-based proteomics has become the gold standard for the large-scale analysis of proteins in complex biological samples. The most widely adopted strategy is the ``bottom-up'' approach. In this workflow, proteins are extracted from a biological source and enzymatically digested—typically using trypsin—into smaller peptides before being analyzed by the mass spectrometer \cite{Gevaert2003}. This enzymatic cleavage is essential because peptides are easier to fractionate, ionize, and fragment than intact proteins. Following digestion, the resulting peptide mixture is usually separated by liquid chromatography (LC) and ionized (e.g., via Electrospray Ionization, ESI) to be transferred into the gas phase for analysis \cite{Aebersold2016}.

The analysis occurs in two primary stages within the mass spectrometer. In the first stage (MS1), the instrument measures the mass-to-charge ratio ($m/z$) and intensity of the intact peptides (precursors). This provides a snapshot of the peptide population in a sample at a given time. From this MS1 information, specific precursor ions are selected for the second stage, tandem mass spectrometry (MS2), to extract structural information for identification.

\subsection{Tandem Mass Spectrometry (MS/MS) and Peptide Fragment Ion Theory}
The identification of the amino acid sequence is achieved through Tandem Mass Spectrometry (MS/MS or MS2). In this process, a specific precursor ion is isolated based on its $m/z$ and subsequently subjected to fragmentation \cite{Steen2004}. In high-resolution instruments like the Orbitrap, Higher-energy Collisional Dissociation (HCD) is the preferred method. HCD is a beam-type collision-induced dissociation technique where ions collide with an inert gas (e.g., Nitrogen), leading to internal energy buildup and subsequent bond breakage, producing a predictable pattern of fragment ions.

According to the Roepstorff-Fohlman nomenclature for peptide fragment ion theory, the fragmentation of the peptide backbone occurs primarily at the amide bonds. In the experimental setting of HCD, this gas-phase dissociation predominantly targets the peptide bonds. This results in two main series of ions: b-ions, where the charge remains on the N-terminal fragment, and y-ions, where the charge remains on the C-terminal fragment \cite{Roepstorff2010}.

\subsection{Tandem Mass Tag (TMT) Labeling}
Tandem Mass Tag (TMT) labeling is a powerful chemical labeling strategy used for high-throughput multiplexed quantitative proteomics. The TMT molecule is an isobaric tag consisting of three functional groups: a reactive NHS-ester group for covalent attachment to peptide N-termini and lysine side chains, a mass reporter group, and a mass normalizer group \cite{Thompson2003}.

Because the tags are isobaric, peptides from different biological samples (up to 18-plex) are labeled, pooled, and appear as a single precursor peak in the MS1 scan \cite{Werner2014}. By enabling the simultaneous analysis of up to 18 samples in a single LC-MS/MS run, TMT labeling minimizes technical batch effects and ensures consistent quantification across channels, reducing the ``missing value'' challenge compared to label-free workflows \cite{Rauniyar2014}. This challenge, common in label-free proteomics, occurs when a peptide is inconsistently detected or fragmented across different runs, leading to incomplete data matrices. In TMT, the co-elution and simultaneous fragmentation of all labeled versions of a peptide ensure that if a signal is detected, quantitative information is typically retrieved for all multiplexed samples. Furthermore, a ``carrier channel''—an isobaric spike-in of a high-abundance proteome—can be used to boost the precursor signal of low-input samples, enhancing identification and sequencing depth in single-cell applications \cite{Budnik2018}.



\subsection{Impact of TMT on Fragmentation Patterns}
The use of TMT tags introduces systematic changes to the peptide fragment spectra. Upon fragmentation via HCD, the isobaric tag cleaves at a specific linker region, releasing the low-molecular-weight reporter ions in the $m/z$ 126–135 range for quantification \cite{McAlister2014}. For identification, TMT labeling presents a challenge: the tag adds a constant mass shift to the N-terminus and lysine residues. Furthermore, the presence of the bulky tag can alter the gas-phase basicity and fragmentation efficiency, often leading to different relative intensities of b- and y-ions compared to unlabeled peptides \cite{Hogrebe2018}.

As we move from the physical process of generating these spectra, the focus shifts to the computational interpretation of this data, which leads to the different identification approaches.

\section{Peptide Identification Strategies}

\subsection{Database Search Engines}
The most prevalent method for peptide identification is database searching. This strategy relies on a predefined protein sequence database, such as UniProt \cite{UniProt2023}. Computational search engines, including Mascot, SEQUEST, or MaxQuant (Andromeda), perform an \textit{in silico} digestion of these sequences to generate a library of theoretical spectra \cite{Cox2008}. Each experimental MS/MS spectrum is then compared against these theoretical candidates using scoring functions to determine the best match, known as a Peptide-Spectrum Match (PSM) \cite{Eng1994}.

While robust, database searching is inherently limited by the predefined sequence search space. Traditional closed searches require an a priori definition of both the protein sequences and their expected modifications. Consequently, peptides from non-model organisms, genomic variants, or those carrying unexpected post-translational modifications (PTMs) are frequently missed. Although open search strategies allow for the identification of PTMs without prior definition by allowing larger precursor mass tolerances, the underlying peptide sequences must still be present in the database. Attempting to account for all possible PTMs within a traditional closed search would lead to a combinatorial explosion, drastically increasing false discovery rates (FDR) and computational costs [Nesvizhskii2010].
\subsection{De Novo Peptide Sequencing}
In contrast to database-driven methods, \textit{de novo} peptide sequencing reconstructs the amino acid sequence directly from the fragment ion peaks in the MS/MS spectrum without any genomic or proteomic reference \cite{Taylor1997}. This approach maps the mass differences between peaks directly to the masses of amino acids. 

By measuring the mass difference between consecutive ions in a series, the corresponding amino acid can be inferred, as each residue (except for the isomers Leucine and Isoleucine) possesses a unique residual mass. For example, a measured mass shift of $113.08$~Da identifies Leucine or Isoleucine, whereas a shift of $71.04$~Da identifies Alanine \cite{Steen2004}. Historically, however, \textit{de novo} sequencing was limited by spectral noise, low resolution, and incomplete fragmentation.

\subsection{Classical De Novo Peptide Sequencing Approaches}
Early \textit{de novo} peptide sequencing algorithms, such as PEAKS, Novor, and PepNovo, are rooted in rule-based modeling of peptide fragmentation \cite{Ma2003, Ma2015Novor, Frank2005}. These methods typically represent an MS/MS spectrum as a spectrum graph, where nodes correspond to observed $m/z$ values and edges represent mass differences matching specific amino acids. Sequence identification is framed as finding the optimal path through this graph, guided by hand-crafted scoring functions that account for peak intensities and ion types.

However, these classical approaches face big limitations. They rely heavily on heuristic fragmentation rules which lack flexibility across different mass spectrometry instruments or chemistries. Incomplete fragmentation often leads to ``broken'' paths in the spectrum graph, while the presence of post-translational modifications (PTMs) or chemical labels like TMT exponentially increases the search space and resulting ambiguity.


\section{Deep Learning and Transformer Models in de novo sequencing}
The limitations of rule-based \textit{de novo} sequencing approaches motivated the adoption of deep learning methods, which learn peptide fragmentation patterns directly from data. By leveraging large annotated MS/MS datasets, neural networks can model complex, instrument-specific fragmentation behavior and generalize across varying experimental conditions. This data-driven paradigm marked a fundamental shift in \textit{de novo} peptide sequencing.

\subsection{The new modeling approach}
The transition to deep learning enabled the modeling of \textit{de novo} sequencing as a sequence-to-sequence (seq2seq) task, translating spectral peak patterns into amino acid sequences. Convolutional Neural Networks (CNNs) were initially used to extract local spectral features, while Recurrent Neural Networks (RNNs), such as Long Short-Term Memory (LSTM) networks, modeled sequential dependencies between amino acids \cite{Tran2017}. Although these models, like DeepNovo, outperformed classical methods in many settings, they exhibited limitations in capturing long-range dependencies and global spectral context, particularly for longer peptides or spectra with sparse fragmentation.

\subsection{Transformer Architecture and Self-Attention}
The introduction of the Transformer architecture \cite{Vaswani2017} addressed many of these limitations by replacing recurrence with self-attention mechanisms. Self-attention allows the model to assess relationships between all spectral features simultaneously, enabling the integration of complementary evidence such as b- and y-ion pairs distributed across the full $m/z$ range.

In proteomics, this capability is especially beneficial, as peptide evidence is often fragmented and non-local. The Casanovo model was a landmark application of Transformers to \textit{de novo} peptide sequencing, employing an encoder–decoder architecture to autoregressively predict peptide sequences \cite{Yilmaz2022}. By combining global spectral context with precursor mass constraints and beam search decoding, Casanovo achieved state-of-the-art performance.

\subsection{Spectrum Encoding and Embedding}
A critical component of Transformer-based models is the encoding of MS/MS spectra into suitable input representations. Typically, spectra are transformed into fixed-length embeddings that incorporate $m/z$ values, intensities, and positional information. These embeddings serve as the input tokens for the Transformer encoder, enabling the model to learn fragmentation-aware representations. Effective spectrum encoding is essential for robust performance, as it directly influences how well the model can distinguish informative peaks from noise and account for variations in fragmentation patterns.

\subsection{Autoregressive Sequence Generation and Beam Search}
DeepNovo pioneered the use of CNNs for feature extraction \cite{Tran2017}. However, the autoregressive decoding strategy, where the model predicts one amino acid at a time conditioned on previously predicted residues and the encoded spectrum, was popularized by Casanovo \cite{Yilmaz2022}. 
Beam search is commonly applied during inference to explore multiple high-probability candidate sequences simultaneously. This approach allows the model to balance local confidence with global sequence plausibility while enforcing constraints such as precursor mass consistency.
As a result, autoregressive decoding with beam search improves identification accuracy, particularly in ambiguous or noisy spectra \cite{Yilmaz2023}.

Following Casanovo, several models have expanded the capabilities of \textit{de novo} sequencing. InstaNovo and $\pi$-PrimeNovo introduce architectural innovations that, among other improvements, increase inference speed. To address the limitation of sparse PTM support in early models, Modanovo extended the token vocabulary to include a broad range of amino acid-PTM combinations, demonstrating that Transformers can scale to biologically diverse datasets \cite{KlaprothAndrade2025}.



